\chapter{力と運動}

\section{力のこと}

物体の運動を変化するとき、または形が変わる時、そこには\term[ちから]{力}が働いている。
力の大きさは\term[にゅうとん]{ニュートン}(記号:$\mathrm{N}$\index{N@N(ニュートン)})という単位で表される。

この節では、身近な力の例を紹介し、それらの作図のしかたを学ぶ。

\subsection{身近な力の例}

\paragraph{重力}
\mbox{}\\
\indent
地球上にある物体は、\term[じゅうりょく]{重力}という力で地球に引っ張られている。
リンゴを手から離すと床に落ちていくのは、この力が働いているからである。
物体がもつ\term[しつりょう]{質量}という量が大きいほど、その物体に働く重力は大きくなる。
質量の大きさは\term[きろぐらむ]{キログラム}(記号:kg\index{kg@kg(キログラム)})という単位で表される。

$1\,\mathrm{kg} (= 1000\,\mathrm{g})$ の物体に加わる重力はおよそ $9.8\,\mathrm{N}$ である。
大雑把に「$100\,\mathrm{g} (= 0.1\,\mathrm{kg}$の物体には $1\,\mathrm{N}$の重力が働く」という感覚を持ってほしい。

\section{「速さ」について改めて}
\index{はやさ@速さ}

例えば次の問題:
\begin{screen}
  小林君は 20~km の距離を2時間かけて走りました。
  小林君は時速何 km で走りましたか。
\end{screen}
を題材として、速さについて復習したい。
この速さの計算で考えるべきことは、「1時間にどれくらいの距離進めるか」という単位量あたりの大きさであった\footnote{%
  謎の語呂合わせを唱えて速さを公式化したり、単位だけを見て「速さは~km/hだから距離を時間で割れば良いんだな」などと安易に考えることを私は推奨しない。
  (初めて速さを学ぶ小学生は特に)速さを割合の問題(具体的には単位量あたりの大きさの問題)に帰着させるべきである。
  「時速何~km」と聞かれているのであれば、1時間にどれくらい進めるかを計算するのだな!と思えばよく、2時間で20~km進めるから1時間ではその半分の10~km進める!と頭を働かせれば良い。
  「時速10~kmの速さで2時間走る時、何~km進めるか」という問題でも、時速10~kmは1時間に10~km進むことを意味しているから、2時間走ればその2倍の距離進めるから20~km進めるぞ!と気づけば良い。
  このような方法は時間のかかるものであるが、下手な公式化により思考停止・わかった気になるという病的な状況を回避するのに有効であり、この先の勉強においても広がりがあり応用の効くものだと考えている。
}。
上の問題の場合、1時間で $20\,\mathrm{km} \div 2 = 10\,\mathrm{km}$進めるから、小林君の速さは時速 $10\,\mathrm{km}$ である。

当然だが、この間小林君は信号で止まったりスパートをかけて加速したりと、一瞬一瞬で速さを変えながら走っている。
先ほど求めた速さは、この途中経過を考慮せず、トータルで走った距離とかかった時間から $\dfrac{\text{距離}}{\text{時間}}$のように計算したものである。
この速さを\term[へいきんのはやさ]{平均の速さ}という。
一方、刻一刻と変わる速さは\term[しゅんかんのはやさ]{瞬間の速さ}と呼ばれる。
瞬間の速さの求め方についてここで詳しく議論しないが、ものすごく短い時間での平均の速さだと思って良い。

次節以降で「速さ」という時、それは「瞬間の速さ」を意味するので注意してほしい。

{\small%
  \noindent
  注意)
  「速さ」は、時速 ◯◯~km のようにどのくらい速いかを表す量である。
  似たような言葉に「速度\index{そくど@速度}」があるが、これはどれくらい速いかという情報に加え、どの向きに進むかという情報をもっている。
  西に時速 10~km で走る西村君と東に 10~km で走る山本君の「速さ」は等しいが「速度」は等しくない。
  本書では(というより中学校の理科では)、物体の運動について学ぶ際、物体の「速さ」のみを取り上げ「速度」について議論はしない。
}
