\chapter{プロローグ}

\paragraph{この本の内容}
\mbox{}\\
\indent
この本では、中学校3年間で学ぶ理科を解説しました。
先取りして中学理科を学びたい小学生が本書だけで独習できるよう、各単元について丁寧で易しい記述を心がけました。
(その副作用として、分量がかなり増えてしまった orz)
もちろん、定期試験や受験に向けて中学理科を復習したい中学生を想定して、各章の最後にはサマリーを設け要点を復習できるようにしました。
もし、サマリーの内容で自分が理解できていないと思う項目が見つかったら、必ず本文の該当箇所を参照するか、学校の教科書を復習するかしてください。
それを怠って、何でもかんでも公式として丸暗記しようとするのは最悪です。
考えることを放棄して(あるいは諦めて)安易に公式化・暗記していては、脳にゴミが積もっていき、脳みそのパンク・崩壊の一途をたどるだけです。
これでは学びに広がりが生まれないだけでなく、試験勉強の方法としても不適切で自分を苦しめてしまいます。
深く理解しようとすればそれなりに時間がかかると思いますが、一杯のコーヒーを相棒に\footnote{%
こんな風に Mr.Children の歌詞を匂わせる記述はこの先の本文にも登場するでしょう。
}じっくり理科を楽しんでほしいです。

\paragraph{サポートページも一応作りました}
\mbox{}\\
\indent
この本に含まれるタイポや内容の誤りは、見つかり次第修正し、
\begin{center}
  \url{https://github.com/sodesudesu/jhScience.git}
\end{center}
に修正版を公開します。

この本には、内容はそのまま文字サイズを変更したバージョンもあります。
それも上のリンクからダウンロードできます。
