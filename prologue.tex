\chapter{プロローグ}

\paragraph{この本の内容}
\mbox{}\\
\indent
この本では、中学校3年間で学ぶ理科を解説した。
先取りして中学理科を学びたい小学生が本書だけで完結して独習できるよう、各単元について丁寧で平易な記述を心がけた。
その副作用として、分量はかなり増えてしまった。
もちろん、定期試験や受験に向けて中学理科を復習したい中学生を想定して、各章の最後にはサマリーを設け要点を復習できるようにした。

考えることを放棄して(あるいは諦めて)なんでも公式化・暗記していては、脳にゴミが積もっていき、脳みそのパンク・崩壊の一途をたどるだけである。
これでは学びに広がりが生まれないだけでなく、試験勉強の方法としても不適切で自分を苦しめてしまう。

\paragraph{サポートページも一応作りました}
\mbox{}\\
\indent
この本に含まれるタイポや内容の誤りは、見つかり次第修正し、
\begin{center}
  \url{https://github.com/sodesudesu/jhScience.git}
\end{center}
に修正版を公開します。

この本には、内容はそのまま文字サイズを変更したバージョンもあります。
それも上のリンクからダウンロードできます。
